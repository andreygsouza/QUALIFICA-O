\chapter{RESULTADOS PRELIMINARES E ESPERADOS}
\label{cap:resultados}

\section{Resultados Preliminares}


Um dos objetivos do presente trabalho é testar a aptidão de diferentes técnicas de processamento e classificação de dados para tarefa de classificação de condutores. Os gráficos da Figura mostram o desempenho em termos da acurácia dos testes juntamente com a dispersão dos resultados dos classificadores, divididos em cada umas das técnicas de redução de dimensionalidade.

Em todos os cenários, o classificador KNN obteve desempenho superior em comparação aos outros classificadores, enquanto o MLP teve um desempenho consideravelmente inferior, além de apresentar valores de desvio padrão superiores, se mostrando uma técnica pouco indicada para a tarefa de identificação de condutores.

Considerando a janela temporal ideal para a identificação de condutores, conclui-se que a janela de 90 segundos maximiza o desempenho do classificador KNN, porém nos casos das técnicas MLP e RF, quanto maior a janela temporal, melhor a taxa de acertos dos mesmos. A janela temporal de 30 segundos apresentou desempenho inferior no KNN e RF, enquanto que no caso do MLP, os valores se estabilizaram a partir da janela de 30 segundo, com acréscimo razoável na janela de 120 segundo. 

Pode-se observar que a janela de 90 segundos tem o melhor desempenho no que diz respeito aos certos quanto à autenticidade do condutor. É valido ressaltar que, em aplicações reais futuras, deve-se determinar uma valor de janela ótimo que alie a rapidez e precisão na identificação do condutor, visto que uma janela de 90 segundo pode ser muito quanto se trata de uma ação criminosa, porém um falso negativo é uma situação que deve ser evitada para a confiabilidade do sistema.

Com relação à eficiência dos métodos de redução de dimensionalidade, nenhumas das técnicas se mostrou superior em questão de desempenho em relação às outras, no qual o objetivo de diminuir o espaço amostral de entrada do classificador foi alcançado sem que se houvesse comprometimento na classificação.

No caso do melhor classificador, o KNN, as três técnicas obtiveram uma quantidade de acertos semelhantes nas mesmas condições de janela temporal. Porém, nos classificadores RF e MLP não houve uma técnica que teve seu desempenho superior em todas as circunstâncias. Os gráficos apresentados na Figura ilustram o desempenho comparado de cada uma das técnicas de redução de dimensionalidade em cada um dos classificadores abordados.

Como o classificador KNN teve um desempenho consideravelmente superior aos demais, optou-se por efetuar uma análise de desempenho mais aprofundada do mesmo, considerando, além da precisão, as métricas de análise de concordância \textit{Cohen's kappa Score} e \textit{F1-Score}. Pode-se observar que em todos os casos, a janela de 90 segundos resulta em um resultado melhor chegando à 99.5\% de acertos com redução PCA e ICA e 99.4\% na redução IPCA.

\begin{table}[htb]
\centering
\caption{Desempenho do Classificador KNN e Redução de Dimensionalidade PCA em diferentes métricas}
\label{tab:desPCA}
\resizebox{0.47\textwidth}{!}{
\begin{tabular}{cccc}
\hline
\multicolumn{4}{c}{PCA} \\ \hline
\multirow{2}{*}{Janela (s)} & \multicolumn{3}{c}{KNN} \\ \cline{2-4} 
 & Acurácia & F1-Score & Cohen Kappa \\ \hline
15 & 0.956 $\pm$ 0 & 0.927 $\pm$ 0 & 0.896 $\pm$ 0 \\
30 & 0.919 $\pm$ 0 & 0.838 $\pm$ 0 & 0.785 $\pm$ 0 \\
60 & 0.971 $\pm$ 0 & 0.951 $\pm$ 0 & 0.930 $\pm$ 0 \\
90 & 0.995 $\pm$ 0 & 0.992 $\pm$ 0 & 0.988 $\pm$ 0 \\
120 & 0.989 $\pm$ 0 & 0.981 $\pm$ 0 & 0.974 $\pm$ 0 \\ \hline
\end{tabular}}
\end{table}

\begin{table}[htb]
\centering
\caption{Desempenho do Classificador KNN e Redução de Dimensionalidade IPCA em diferentes métricas}
\label{tab:desIPCA}
\resizebox{0.47\textwidth}{!}{
\begin{tabular}{cccc}
\hline
\multicolumn{4}{c}{IPCA} \\ \hline
\multicolumn{1}{l}{\multirow{2}{*}{Janela (s)}} & \multicolumn{3}{c}{KNN} \\ \cline{2-4} 
\multicolumn{1}{l}{} & \multicolumn{1}{l}{Acurácia} & \multicolumn{1}{l}{F1-Score} & \multicolumn{1}{l}{Cohen Kappa} \\ \hline
15 & 0.956 $\pm$ 0 & 0.927 $\pm$ 0 & 0.896 $\pm$ 0 \\
30 & 0.922 $\pm$ 0 & 0.842 $\pm$ 0 & 0.7905 $\pm$ 0 \\
60 & 0.972 $\pm$ 0 & 0.952 $\pm$ 0 & 0.932 $\pm$ 0 \\
90 & 0.994 $\pm$ 0 & 0.990 $\pm$ 0 & 0.986 $\pm$ 0 \\
120 & 0.989 $\pm$ 0 & 0.981 $\pm$ 0 & 0.974 $\pm$ 0 \\ \hline
\end{tabular}}
\end{table}

\begin{table}[htb]
\centering
\caption{Desempenho do Classificador KNN e Redução de Dimensionalidade ICA em diferentes métricas}
\label{tab:desICA}
\resizebox{0.47\textwidth}{!}{
\begin{tabular}{cccc}
\hline
\multicolumn{4}{c}{ICA} \\ \hline
\multirow{2}{*}{Janela (s)} & \multicolumn{3}{c}{KNN} \\ \cline{2-4} 
 & Acurácia & F1-Score & Cohen Kappa \\ \hline
15 & 0.957 $\pm$ 0 & 0.927 $\pm$ 0 & 0.896 $\pm$ 0 \\
30 & 0.917 $\pm$ 0 & 0.834 $\pm$ 0 & 0.778 $\pm$ 0 \\
60 & 0.976 $\pm$ 0 & 0.960 $\pm$ 0 & 0.943 $\pm$ 0 \\
90 & 0.995 $\pm$ 0 & 0.991 $\pm$ 0 & 0.988 $\pm$ 0 \\
120 & 0.988$ \pm$ 0 & 0.980 $\pm$ 0 & 0.971 $\pm$ 0 \\ \hline
\end{tabular}}
\end{table}


\section{Resultados esperados}

Espera-se com os resultados desta pesquisa, desenvolver um sistema, que, através de técnicas de inteligência computacional, possa identificar, autenticar e discernir condutores por meio de sinais presentes no barramento CAN e \textit{smartphones}. Para tanto, como produto final desta pesquisa, espera-se o desenvolvimento de um aplicativo móvel que desempenhe as tarefas mencionadas, com uma alta taxa de acertos na autenticação dos condutores. Estes resultados poderão contribuir para a implementação futura em larga escala do sistema em \textit{hardware} embarcado no veículo, podendo contribuir para a mitigação de roubos e furtos de veículos, além de outras aplicações. Também espera-se a publicação de um ou mais artigos ou resumos expandidos em periódicos ou congressos da área de Engenharias IV.