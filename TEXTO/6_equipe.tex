\chapter{EQUIPE}
\label{cap:equipe}

A equipe de pesquisadores que irá trabalhar na pesquisa e desenvolvimento no presente projeto está descrito a seguir:

\begin{itemize}
	\item Dr. Wilian Soares Lacerda (Orientador): é Técnico em Eletrônica pelo Centro Centro Federal de Educação Tecnológica de Minas Gerais - CEFET/MG (1986), possui graduação em Engenharia Elétrica pela Universidade Federal de Minas Gerais (1991), mestrado em Engenharia Elétrica (área Automática) pela Universidade Federal de Minas Gerais (1994) e doutorado em Engenharia Elétrica (linha Engenharia da Computação) pela Universidade Federal de Minas Gerais (2006). Atualmente é professor associado da Universidade Federal de Lavras atuando no Departamento de Ciência da Computação onde leciona as disciplinas de graduação e pós-graduação: Eletrônica Básica, Sistemas Digitais, Sistemas Embarcados e Microcontroladores, Redes Neurais Artificiais. Desenvolve pesquisa na área de Inteligência Computacional, atuando principalmente nos seguintes temas: Redes Neurais Artificiais, Sistemas Fuzzy e Computação Evolutiva. Desenvolve protótipos de sistemas embarcados para aplicações específicas utilizando: microcontroladores, \textit{hardware} reconfigurável (FPGA), sensores e atuadores diversos.
	
	\item Andrey Gustavo de Souza (Mestrando): possui graduação em Engenharia Elétrica pelo Instituto Federal de Minas Gerais (IFMG) - Campus Formiga. É aluno de mestrado em Engenharia de Sistemas e Automação, na linha de pesquisa em Sistemas Inteligentes, pela Universidade Federal de Lavras (UFLA). É ex-bolsista do programa Ciência sem Fronteiras nos Estados Unidos, tendo feito imersão de inglês na State University of New York (SUNY), Nassau Community College e programa acadêmico na faculdade de engenharia Elétrica da SUNY Maritime College, participando de projetos de pesquisa com o corpo docente da instituição. Atuou como bolsista em projetos de Iniciação Científica dentro do IFMG - Campus Formiga. Têm interesse nas áreas de Sistemas Embarcados, Sistemas Automotivos, Inteligência Computacional e Processamento de Dados.
\end{itemize}

Poderá-se avaliar a necessidade da inclusão de outros membros à equipe de pesquisa, dentre coorientadores e alunos de iniciação científica, de acordo com as demandas identificadas no andamento do projeto.