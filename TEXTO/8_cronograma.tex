\chapter{CRONOGRAMA}
\label{cap:crono}

Com início no primeiro período de 2017 (maio), o programa de Mestrado se iniciou com a escolha do tema da dissertação, sendo definido o título do presente projeto de pesquisa. Juntamente com o orientador, foram definidas as disciplinas que seriam cursadas. No primeiro período de 2017 foram cursadas as disciplinas: PCH501 Inglês Instrumental, PSI505 Tópicos Especiais em Engenharia de Sistemas e Automação (Redes Neurais Artificiais), PSI515 Pesquisa Bibliográfica e Comunicação Científica, PSI528 Processamento de Sinais e PSI531 Sistemas Fuzzy. No segundo período de 2017 foram cursadas as seguintes disciplinas: PSI504 Seminário I, PSI511 Estágio Docência I MS, PSI514 Projeto Orientado, PSI516 Computação Evolucionária, PSI519 Análise de Componentes Independentes e PSI535 Eletrônica de Potência Aplicada a Sistemas Elétricos. Estas disciplinas integralizam a carga horária mínima obrigatória de 24 créditos. 

Dando continuidade ao trabalho que resultará no dissertação de mestrado, é determinado o cronograma apresentado na Tabela~\ref{tab:crono}, com as seguintes atividades:

\begin{enumerate}
    \item Revisão da literatura, montagem de acervo bibliográfico e análise do estado da arte do sistema proposto.
    
    \item Escrita do Projeto de Qualificação.
    
    \item Exame de Qualificação.
    
    \item Aquisição de dados preliminares para desenvolvimento do sistema.
    
    \item Avaliação e implementação de métodos de processamento de dados e algoritmo de identificação de condutores, por meio do \textit{dataset} UYANIK e dados colhidos preliminarmente.
    
    \item Construção de \textit{benchmark} comparativo entre as diversas possibilidades de técnicas de processamento e identificação e escolha dos métodos a serem implementados.
    
    \item Implementação de aplicativo Android que irá desempenhar o classificador.
    
    \item Teste e validação do sistema proposto, por meio de experimentos práticos e avaliação dos resultados.
    
    \item Escrita da dissertação.
    
    \item Defesa.
    
\end{enumerate}

\begin{table}[!htb]
	\centering
	\caption{Cronograma das atividades que serão realizadas.}
	\label{tab:crono}
\resizebox{\textwidth}{!}{\begin{tabular}{|c|c|c|c|c|c|c|c|c|c|c|c|c|c|c|c|c|}
\hline
\multirow{2}{*}{Atividade} & \multicolumn{4}{c|}{2017} & \multicolumn{12}{c|}{2018} \\ \cline{2-17} 
 & \multicolumn{1}{l|}{Set.} & \multicolumn{1}{l|}{Out.} & \multicolumn{1}{l|}{Nov.} & \multicolumn{1}{l|}{Dez.} & \multicolumn{1}{l|}{Jan.} & \multicolumn{1}{l|}{Fev.} & \multicolumn{1}{l|}{Mar.} & \multicolumn{1}{l|}{Abr.} & \multicolumn{1}{l|}{Maio} & \multicolumn{1}{l|}{Jun.} & \multicolumn{1}{l|}{Jul.} & \multicolumn{1}{l|}{Ago.} & \multicolumn{1}{l|}{Set.} & \multicolumn{1}{l|}{Out.} & \multicolumn{1}{l|}{Nov.} & \multicolumn{1}{l|}{Dez.} \\ \hline
1 & X & X & X &  &  &  &  &  &  &  &  &  &  &  &  &  \\ \hline
2 &  & X & X & X & X &  &  &  &  &  &  &  &  &  &  &  \\ \hline
3 &  &  &  &  &  & X &  &  &  &  &  &  &  &  &  &  \\ \hline
4 &  &  &  &  & X & X & X &  &  &  &  &  &  &  &  &  \\ \hline
5 &  &  &  &  &  &  & X & X & X &  &  &  &  &  &  &  \\ \hline
6 &  &  &  &  &  &  &  &  & X & X & X &  &  &  &  &  \\ \hline
7 &  &  &  &  &  &  &  &  &  & X & X & X & X &  &  &  \\ \hline
8 &  &  &  &  &  &  &  &  &  &  &  &  & X & X & X &  \\ \hline
9 &  &  &  &  &  &  &  &  & X & X & X & X & X & X & X &  \\ \hline
10 &  &  &  &  &  &  &  &  &  &  &  &  &  &  &  & X \\ \hline
\end{tabular}}
\centering {\small Fonte: Próprio autor.} %Fonte do quadro
\end{table}