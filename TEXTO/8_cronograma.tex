\chapter{CRONOGRAMA}
\label{cap:crono}

Com início no primeiro período de 2017 (maio), o programa de Mestrado se iniciou com a escolha do tema da dissertação, sendo definido o título do presente projeto de pesquisa. Juntamente com o orientador, foram definidas as disciplinas que seriam cursadas. No primeiro período de 2017 foram cursadas as disciplinas: PCH501 Inglês Instrumental, PSI505 Tópicos Especiais em Engenharia de Sistemas e Automação (Redes Neurais Artificiais), PSI515 Pesquisa Bibliográfica e Comunicação Científica, PSI528 Processamento de Sinais e PSI531 Sistemas Fuzzy. No segundo período de 2017 foram cursadas as seguintes disciplinas: PSI504 Seminário I, PSI511 Estágio Docência I MS, PSI514 Projeto Orientado, PSI516 Computação Evolucionária, PSI519 Análise de Componentes Independentes e PSI535 Eletrônica de Potência Aplicada a Sistemas Elétricos. Estas disciplinas integralizam a carga horária mínima obrigatória de 24 créditos. 

Dando continuidade ao trabalho que resultará no dissertação de mestrado, é determinado o cronograma apresentado na Tabela~\ref{tab:crono}, com as seguintes atividades:

\begin{enumerate}
	\item Início do mestrado.
	
	\item Cumprimento da carga horária mínima exigida em disciplinas.
	
    \item Revisão da literatura, construção de acervo bibliográfico e análise do estado da arte do sistema proposto.
    
    \item Escrita do Projeto de Qualificação.
    
    \item Avaliação e implementação de testes preliminares de métodos de processamento de dados e algoritmo de identificação de condutores.
    
    \item Avaliação da viabilidade do sistema de identificação de condutores por meio dos resultados preliminares.
    
    \item Exame de Qualificação.
    
    \item Avaliação de técnicas de aprendizado \textit{online} a serem implementadas.
    
    \item Implementação e testes com as técnicas escolhidas e definição da técnica a ser utilizada no sistema.
    
    \item Teste e validação do sistema proposto, através de experimentos e avaliação dos resultados.
    
    \item Escrita da dissertação.
    
    \item Defesa da dissertação.
    
\end{enumerate}

\begin{table}[h]
	\centering
	\caption{Cronograma das atividades que serão realizadas.}
	\label{tab:crono}
\resizebox{\textwidth}{!}{\begin{tabular}{|c|c|c|c|c|c|c|c|c|c|c|c|c|c|c|c|c|c|c|c|c|c|c|}
		\hline
		\multirow{2}{*}{Atividade} & \multicolumn{8}{c|}{2017} & \multicolumn{12}{c|}{2018} & \multicolumn{2}{c|}{2019} \\ \cline{2-23} 
		& Maio & Jun. & Jul. & Ago. & Set. & Out. & Nov. & Dez. & Jan. & Fev. & Mar. & Abr. & Maio & Jun. & Jul. & Ago. & Set. & Out. & Nov. & Dez. & Jan. & Fev. \\ \hline
		1 & X &  &  &  &  &  &  &  &  &  &  &  &  &  &  &  &  &  &  &  &  &  \\ \hline
		2 & X & X & X & X & X & X & X & X & X & X &  &  &  &  &  &  &  &  &  &  &  &  \\ \hline
		3 &  &  &  &  & X & X & X & X & X & X & X &  &  &  &  &  &  &  &  &  &  &  \\ \hline
		4 &  &  &  &  &  &  & X & X & X & X & X & X &  &  &  &  &  &  &  &  &  &  \\ \hline
		5 &  &  &  &  &  & X & X & X & X & X & X &  &  &  &  &  &  &  &  &  &  &  \\ \hline
		6 &  &  &  &  &  &  &  &  &  &  &  & X &  &  &  &  &  &  &  &  &  &  \\ \hline
		7 &  &  &  &  &  &  &  &  &  &  &  &  & X &  &  &  &  &  &  &  &  &  \\ \hline
		8 &  &  &  &  &  &  &  &  &  &  &  &  &  & X & X & X &  &  &  &  &  &  \\ \hline
		9 &  &  &  &  &  &  &  &  &  &  &  &  &  &  &  & X & X & X &  &  &  &  \\ \hline
		10 &  &  &  &  &  &  &  &  &  &  &  &  &  &  &  &  &  & X & X & X &  &  \\ \hline
		11 &  &  &  &  &  &  &  &  &  &  &  &  &  &  & X & X & X & X & X & X & X &  \\ \hline
		12 &  &  &  &  &  &  &  &  &  &  &  &  &  &  &  &  &  &  &  &  &  & X \\ \hline
\end{tabular}}
\centering {\small Fonte: Próprio autor.} %Fonte do quadro
\end{table}